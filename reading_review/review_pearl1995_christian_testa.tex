\documentclass[a4paper,11pt]{article} 

\usepackage[
  top = .5in, 
  bottom = .5in, 
  left = .75in, 
  right = .75in]{geometry} 

\usepackage[T1]{fontenc}
\usepackage[utf8]{inputenc}
\usepackage{multirow} 
\usepackage{booktabs} 
\usepackage{graphicx} 

\usepackage{setspace}
\setlength{\parindent}{0in}

\usepackage{float}
\usepackage{fancyhdr}

%%%%% header %%%%%

\pagestyle{fancy} 
\fancyhf{} 

\lhead{\footnotesize BST 258: Reading Review 2}% 
\rhead{\footnotesize Christian Testa} 
\cfoot{\footnotesize \thepage} 

%%%%%%%%%%%%%%%%%%%%%%%%%%%%%%%%%%%%%%%%%%%%%%%%
% begin document
%%%%%%%%%%%%%%%%%%%%%%%%%%%%%%%%%%%%%%%%%%%%%%%%

\begin{document}

\thispagestyle{empty} % disable header on the first page. 

\begin{tabular}{p{6.5in}}  
{\bf BST 258: Causal Inference: Theory and Practice} \\
Christian Testa\\ February 16th, 2024\\
Reading Review 1\\
\hline 
\end{tabular} 

\nocite{*}

\vspace*{0.25in}

%%%% body %%%% 

\underline{Summary:} For this week, we were asked to read Judea Pearl's 1995 paper \textit{Causal Diagrams for Empirical Research} published in Biometrika. This article introduces
now-classic concepts such as $d$-separation, graph representations of 
causal dependency among variables, the `back-door' criterion, and the 
`front-door' criterion. Pearl walks the reader through some applied
examples such as how confounding is represented in a graph notation, and 
graphical tests for identifiability of causal effects.  Some of the 
material that was new-er to me was his commentary on how statisticians
had historically viewed graphs with ``suspicion,'' which I thought 
was rather interesting given that SEM was developed in the early 1900s 
with applications to genetic inheritance and made ample use of graphs.

\vspace{0.1in}

\underline{Reflections:} In a way, it is particularly challenging to write 
about how this paradigm and paper relate to the causal inference theory 
with which I am familiar because I feel that this way of thinking 
is foundational and integrated into much of modern causal inference.
Where I find this body of theory somewhat challenging is in 
the specification of the non-parametric functional relationships
between variables because, in general, I find it much more difficult
to show that one variable \textit{cannot} depend on another than to reason 
that one variable \textit{could} depend on another.  As a result, we end up 
with quite ridiculous looking causal graphs where everything forward
in time depends on everything prior in time, and it's not clear 
that this gains us much in our formal thinking. What I do think is 
a boon to the theory of causal inference is how we can bring in 
ideas from graph-theory and computer science to help us reason about
the structure of our data and the relationships between variables.

\end{document}
